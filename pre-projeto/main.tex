\documentclass[12pt, a4paper, onecolumn]{exam}
\usepackage{amsmath}
\usepackage{amssymb}
\usepackage[lmargin=71pt, tmargin=1.2in]{geometry}  %For centering solution box
\usepackage{graphicx} % Required for inserting images
\usepackage{lastpage}
\usepackage{qrcode}
\usepackage{hyperref}

% Document's meta-data
\newcommand{\subject}{Proposta de Tema de TCC}
\newcommand{\authorfullname}{José Augusto Queiroz Comparotto Gomes}
\newcommand{\authorname}{José A. Q. C. Gomes}
\newcommand{\authorno}{398439413098}
\newcommand{\professor}{Prof.ª Noiza Waltrick Trindade}
\newcommand{\course}{Ciência da Computação — Noturno}
\newcommand{\location}{Campo Grande-MS}
\newcommand{\documentdate}{1º de outubro de 2024}
\newcommand{\university}{Universidade Anhaguera — Uniderp}

\author{\authorfullname}
\title{\subject}

% General page styling
\lhead{\subject\\}
\lfoot{\authorname}
\cfoot{Página \thepage \ de \pageref{LastPage}}
\rfoot{RA: \authorno}

\thispagestyle{empty}   %For removing header/footer from page 1

\begin{document}

% Presenting page
\begingroup  

    \centering
    \begin{figure}
        \centering
        \includegraphics[width=0.15\linewidth]{assets/uniderp.jpg}
        \label{fig:university-logo}
    \end{figure}
    
    \rule{\textwidth}{2pt}  \\[1em]
    
    \LARGE \subject     \\
    
    \vfill
    
    \large À \professor \\

    \vfill
    
    \large \textbf{Título Proposto:} \\[1em]
    \LARGE Arquitetura e Estratégias para Integração de Sistemas de Informação Gerenciais: Desafios e Soluções para Softwares Proprietários
    
    \vfill

    \large \textbf{Acadêmico:} \\[1em]
    \LARGE \authorfullname
    
    \vfill
    
    \large \course          \\
    \large RA: \authorno    \\[1em]
    
    \rule{\textwidth}{2pt}  \\[1em]

    \large \location        \\
    \large \documentdate    \\

    \pagebreak
\endgroup

\section{Introdução e Contextualização}

No cenário atual de negócios, sistemas de informação gerenciais (SIG's) desempenham um papel crucial na tomada de decisões e no gerenciamento de recursos. No entanto, a integração de diferentes sistemas de informação, especialmente em ambientes corporativos que utilizam softwares proprietários, apresenta uma série de desafios técnicos e arquiteturais.

Softwares proprietários, por sua natureza, muitas vezes não seguem padrões abertos ou práticas de desenvolvimento que facilitem a interoperabilidade. Essa falta de conformidade pode causar dificuldades significativas na integração entre sistemas, exigindo soluções criativas e técnicas para superar essas barreiras.

\section{Objetivo Geral}

Este trabalho procura investigar as características arquiteturais e técnicas desejáveis em um sistema de informação gerencial que facilite sua integração com outros sistemas. Ao longo da pesquisa, serão discutidos os principais obstáculos encontrados em softwares proprietários e apresentadas alternativas técnicas para viabilizar a integração, mesmo em cenários adversos.

\section{Objetivos Específicos}

\begin{itemize}
    \item Identificar e descrever boas práticas arquiteturais que tornam um sistema mais amigável para integrações.

    \item Analisar problemas comuns encontrados em softwares que não seguem essas práticas, explicando as dificuldades que surgem no processo de integração.

    \item Apresentar soluções e alternativas para integrar sistemas legados ou proprietários que não implementam padrões abertos.

    \item Realizar um estudo de caso hipotético ou prático, demonstrando a aplicação das soluções propostas.

\end{itemize}

\section{Justificativa}

A integração entre sistemas de informação é uma necessidade crescente em organizações de todos os portes. No entanto, a falta de padrões e a rigidez de softwares proprietários dificultam a comunicação entre diferentes plataformas. Ao abordar esses desafios de forma teórica e prática, este trabalho contribuirá para o campo de Sistemas de Informação Gerenciais, fornecendo orientações para a criação de sistemas mais integráveis e, ao mesmo tempo, sugerindo soluções para os casos em que essas boas práticas não são seguidas.

\section{Metodologia}

O trabalho será de natureza exploratória e teórica, com foco em:
\begin{enumerate}
    \item[1.] Revisão bibliográfica sobre boas práticas de integração de sistemas, com ênfase em APIs RESTful, modularidade, modelagem de dados, e estratégias de autenticação.
    
    \item[2.] Análise de problemas técnicos comuns enfrentados na integração com sistemas proprietários e propostas de soluções baseadas em arquitetura de software e padrões de integração.
    
    \item[3.] Desenvolvimento de um estudo de caso que simule a integração de sistemas com diferentes características, demonstrando os desafios e soluções abordadas.
\end{enumerate}
\section{Considerações Finais}

\sloppy
Esta proposta visa explorar um campo relevante no contexto empresarial atual, trazendo contribuições tanto para a academia quanto para a prática profissional na área de integração de sistemas de informação gerenciais. O objetivo é fornecer orientações para o desenvolvimento de sistemas mais amigáveis à integração, ao mesmo tempo que aborda soluções práticas para integrar softwares proprietários que apresentam desafios específicos.

\end{document}
